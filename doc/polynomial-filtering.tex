\documentclass[%
	paper=a4,
	fontsize=10pt,
	DIV11,BCOR10mm,
	numbers=noenddot,
	abstract=yes
]{scrartcl}

\usepackage[utf8]{inputenc}
\usepackage[english]{babel}
\usepackage[T1]{fontenc}
\usepackage{tgpagella}
\usepackage{microtype}


\usepackage{textcomp}
\usepackage{gensymb}


\usepackage{graphicx}
\graphicspath{{figures/}}


\usepackage{xfrac}
\usepackage{units}


\usepackage{amsmath}
\usepackage{amssymb}
\usepackage{amsthm}


\usepackage{mathtools}
\mathtoolsset{showonlyrefs=true}


\usepackage{scrhack} % for algorithm package
\usepackage[boxed]{algorithm}
\usepackage{algpseudocode}


\usepackage[
	style=alphabetic,
	bibstyle=alphabetic,
	maxbibnames=100,
	giveninits=true,
	useprefix=true,
	natbib=true,
	backend=biber]{biblatex}
\usepackage[strict=true]{csquotes}
\addbibresource{references.bib}


\usepackage{hyperref}
\hypersetup{
	%colorlinks,
	%citecolor=black,
	%filecolor=black,
	%linkcolor=black,
	%urlcolor=black,
	pdfauthor={Christoph Conrads},
	unicode=true,
}


% Make links jump to the beginning of figures and tables, not to the caption in
% them.
% This can also be fixed with the caption package.
\usepackage[all]{hypcap}



% bibliography with ngerman: @mastersthesis != Magisterarbeit
\DefineBibliographyStrings{english}{%
	mathesis = {MSc thesis},
}
\DefineBibliographyStrings{ngerman}{%
	mathesis = {Masterarbeit},
}



% Command definitions
\newcommand{\bigOh}[1]{\mathcal{O}(#1)}
\newcommand{\cond}[2][]{\kappa_{#1}(#2)}
\newcommand{\function}[1]{\textsc{#1}}
\newcommand{\almost}[1]{\widetilde{#1}}
\newcommand{\set}[1]{\mathcal{#1}}

\newcommand{\R}{\mathbb{R}}
\newcommand{\F}{\mathbb{C}}

\newcommand{\macheps}{\varepsilon}
\newcommand{\unitRO}{\mathbf{u}}
\newcommand{\tol}{\operatorname{tol}}

\DeclarePairedDelimiter\abs{\lvert}{\rvert}
\DeclarePairedDelimiter\norm{\lVert}{\rVert}
\DeclarePairedDelimiter\ceil{\lceil}{\rceil}

\DeclareMathOperator{\diag}{diag}
\DeclareMathOperator{\ran}{ran}
\DeclareMathOperator{\rank}{rank}
\DeclareMathOperator{\mathspan}{span}



\newtheorem{theorem}{Theorem}[section]
\newtheorem{corollary}{Corollary}[section]

\theoremstyle{definition}
\newtheorem{example}{Example}[section]
\newtheorem{definition}{Definition}[section]

% \bordermatrix with custom delimiters
% Source: Herbert Voß: "Math mode - v. 2.47", 2014, §5
\makeatletter
\newif\if@borderstar
\def\bordermatrix{\@ifnextchar*{%
    \@borderstartrue\@bordermatrix@i}{\@borderstarfalse\@bordermatrix@i*}%
}
\def\@bordermatrix@i*{\@ifnextchar[{\@bordermatrix@ii}{\@bordermatrix@ii[()]}}
\def\@bordermatrix@ii[#1]#2{%
\begingroup
  \m@th\@tempdima8.75\p@\setbox\z@\vbox{%
    \def\cr{\crcr\noalign{\kern 2\p@\global\let\cr\endline }}%
    \ialign {$##$\hfil\kern 2\p@\kern\@tempdima & \thinspace %
    \hfil $##$\hfil && \quad\hfil $##$\hfil\crcr\omit\strut %
    \hfil\crcr\noalign{\kern -\baselineskip}#2\crcr\omit %
    \strut\cr}}%
  \setbox\tw@\vbox{\unvcopy\z@\global\setbox\@ne\lastbox}%
  \setbox\tw@\hbox{\unhbox\@ne\unskip\global\setbox\@ne\lastbox}%
  \setbox\tw@\hbox{%
    $\kern\wd\@ne\kern -\@tempdima\left\@firstoftwo#1%
      \if@borderstar\kern2pt\else\kern -\wd\@ne\fi%
    \global\setbox\@ne\vbox{\box\@ne\if@borderstar\else\kern 2\p@\fi}%
    \vcenter{\if@borderstar\else\kern -\ht\@ne\fi%
      \unvbox\z@\kern-\if@borderstar2\fi\baselineskip}%
      \if@borderstar\kern-2\@tempdima\kern2\p@\else\,\fi\right\@secondoftwo#1 $%
  }\null \;\vbox{\kern\ht\@ne\box\tw@}%
\endgroup
}
\makeatother

\newcommand{\mybordermatrix}[1]{\bordermatrix[{[]}]{#1}}



\renewcommand{\algorithmicrequire}{\textbf{Input:}}
\renewcommand{\algorithmicensure}{\textbf{Output:}}
\algrenewcommand{\algorithmiccomment}[1]{\hskip1em\# #1}



\newcommand{\propername}[1]{\textsc{#1}}





\title{Polynomial Filtering}
\author{Christoph Conrads {\small \url{https://christoph-conrads.name}}}



\begin{document}

\maketitle



\section{Motivation}

DCGeig computes all eigenpairs $(\lambda, x)$, of a generalized eigenvalue
problem $Kx = \lambda Mx$, where $\lambda \leq \lambda_c$, $\lambda_c > 0$, $K,
M \in \F^{n,n}$ are Hermitian positive definite (HPD). Assume $K, M$ are
partitioned conformally into $2 \times 2$ block matrices:
\[
	\begin{pmatrix} K_{11} & K_{12} \\ K_{21} & K_{22} \end{pmatrix}
	x
	= \lambda \begin{pmatrix} K_{11} & K_{12} \\ K_{21} & K_{22} \end{pmatrix}x.
\]
Broadly speaking, DCGeig applies the following procedure recursively:
\begin{itemize}
	\item heuristically determine a cutoff value $\lambda_c' > \lambda_c$,
	\item compute all eigenpairs $(\lambda_i, x_i)$ of the subproblems
		$K_{ii} x_i = \lambda_i M_{ii} x_i$, $i = 1, 2$, where
		$\lambda_i \leq \lambda_c'$,
	\item lift $x_i$, $i = 1, 2$, to $F^n$ and use these lifted vectors as
		a search space for the exact eigenvectors of $(K, M)$,
	\item improve the search space,
	\item compute eigenpairs $(\lambda, x)$ of $(K, M)$,
	\item select eigenpairs $\lambda \leq \lambda_c$.
\end{itemize}
The first problem with this approach is the presence of $\lambda_c'$: it cannot
be chosen accurately a priori and if $\lambda_c'$ is too large, then the number
of vectors spanning the search space increases and the solver will be slowed
down or run out of memory; if $\lambda_c'$ is too small, then the search space
for the desired eigenvectors of $(K, M)$ shrinks and the solver will be slowed
down\footnote{Let $s$ be the dimension of the search space and let $\lambda_1
\leq \lambda_2 \leq \dotsb \leq \lambda_n$. The solver uses subspace iterations
[SI] to improve the search space and the convergence rate of SI is directly
related to the ratio $\sfrac{\lambda_c}{\lambda_s}$ - the lower the better
\cite[§5.2]{Saad2011}. Consequently, if $s$ decreases, then
$\sfrac{\lambda_c}{\lambda_s}$ increases.} or miss eigenpairs. The second
problem is the computation of exact (sufficiently accurate) eigenpairs in the
subproblems because what we want to achieve with the recursion is to find a
search space that is large enough to accomodate all desired eigenvectors of $(K,
M)$. Instead, we select the search space based on the eigenvalues $\lambda_i$
and the heuristically selected value $\lambda_c'$.

A faster method yielding a good search space might work as follows:
\begin{itemize}
	\item we compute the number $n_c$ of desired eigenpairs $(\lambda, x)$ of $(K, M)$,
	\item we choose positive integers $n_1$, $n_2$ such that $n_1+n_2 \geq n_c$,
		and
	\item we calculate $n_i$ linearly independent vectors that span the space of
		eigenvectors $x_i$ corresponding to the smallest eigenvalues of
		$(K_{ii}, M_{ii})$, $i = 1, 2$.
\end{itemize}
The question arises, if there are fast methods for the computation (or
estimation) of $n_c$ and we can answer the question in the affirmative. Since
$K$ and $M$ are Hermitian, we can apply Sylvester's law of inertia
\cite[Theorem~4.5.8]{Golub2012}, compute the $LDL^T$ decomposition of $K -
\lambda_c M$, and count the number of non-positive diagonal entries to $D$ to
calculate $n_c$. Algorithms for the sparse $LDL^T$ decomposition are implemented
in MUMPS\footnote{\url{http://mumps.enseeiht.fr/}},
PaStiX\footnote{\url{http://pastix.gforge.inria.fr/}},
SPOOLES\footnote{\url{http://www.netlib.org/linalg/spooles/}},
SPRAL\footnote{\url{http://www.numerical.rl.ac.uk/spral/}}, and
TAUCS\footnote{\url{http://www.tau.ac.il/~stoledo/taucs/}} (SPOOLES and TAUCS are
unmaintained). The major drawback of this approach is the memory consumption due
to the fill-in.

Alternatively, we can use a combination of random sampling and polynomial
filtering to estimate $n_c$ and this approach has been thoroughly analyzed in
\cite{DiNapoli2016}. Rational expansion filtering will not be discussed here
because the author is more familiar with polynomial filtering, because rational
filtering seems to be better suited when it is combined with an iterative solver
(rational filtering requires the factorization of $K - \sigma M$ for different
shifts $\sigma$), and because we would need software for the factorization of
indefinite, Hermitian matrices $K - \sigma M$, i.\,e., we would end up computing
$LDL^T$ decompositions. Rational filtering is already used in practice, e.\,g.,
see \cite{Polizzi2009}.



\section{Estimating Eigenvalue Counts}

Let $A \in \F^{n,n}$ with entries $a_{ij}$, $i,j = 1, 2, \dotsc, n$. Throughout
this text, we assume that $A$ has real eigenvalues.

We need the following theorem about the trace
\[ \operatorname{tr}(A) = \sum_{i=1}^n a_{ii}. \]

\begin{theorem}[{\cite[§2.4.1]{Horn2012}}]
	Let $A \in \F^{n,n}$. Then the trace of $A$ is equal to the sum of its
	eigenvalues:
	\[ \operatorname{tr}(A) = \sum_{i=1}^n \lambda_i. \]
\end{theorem}

The concept of projection will be essential for our purposes.

\begin{definition}[{\cite[§0.9.13]{Horn2012}}]
	Let $P \in \F^{n,n}$. If $P = P^2$, then $P$ is called a \emph{projection}.
\end{definition}

Depending on the literature, $P$ may also be called a \emph{projector}, e.\,g.,
in \cite{Polizzi2009}.

\begin{theorem}
	Let $P \in \F^{n,n}$ be a projection. Then $P$ can only have the eigenvalue
	zero and one.
\end{theorem}

\begin{proof}
	Let $J \in \F^{n,n}$ be the Jordan canonical form of $P$
	\cite[§3.1]{Horn2012}. Since $P = P^2$, it must hold that $J = J^2$ so $J$
	is diagonal. Also, since $J = J^2$, all values on the diagonal of $J$ must
	be one or zero. The diagonal entries of $J$ are the eigenvalues of $P$ and
	this completes the proof (NB: This solves \cite[3.3P3]{Horn2012} partially.)
\end{proof}

The identity matrix is a projection whose spectrum contains only ones whereas
the zero matrix has only the eigenvalue zero. We will now show one way to
construct projections.

\begin{definition}[Isometric Matrix]
	Let $Q \in \F^{n,m}$, $m \leq n$, let $I_m$ be the $m \times m$ identity
	matrix. If $Q^* Q = I_m$, then $Q$ is called \emph{isometric}.
\end{definition}

$Q$ may also be called \emph{Euclidean isometry}
\cite[Definition~2.1.5]{Horn2012}. Note that every column subset of a unitary
matrix forms an isometric matrix.

\begin{theorem}
	Let $Q \in \F^{n,m}$ be isometric. Then $Q Q^*$ is a projection.
\end{theorem}

\begin{proof}
	Let $P = Q Q^*$. It holds that
	\[ P^2 = Q Q^* Q Q^* = Q I_m Q^* = Q Q^* = P. \]
\end{proof}

\begin{theorem}
	Let $Q \in \F^{n,m}$ be isometric. Then the projection $P = Q Q^*$ has the
	eigenvalue one with algebraic multiplicity $m$ and the eigenvalue zero with
	algebraic multiplicity $n - m$.
\end{theorem}

\begin{proof}
	From $Q^* Q = I_m$, it follows that the columns of $Q$ are orthonormal. Let
	$Q = [q_1, q_2, \dotsc, q_m]$ be a column-partitioning of $Q$. Then $P = Q
	Q^*$ can be written as $P = \sum_{i=1}^n q_i q_i^*$. Consequently, $P$ is a
	Hermitian matrix with rank $m$ and eigenvalues one and zero.
\end{proof}

\begin{corollary}
	Let $Q \in \F^{n,m}$ be isometric, let $P = Q Q^*$. Then
	\[ \operatorname{tr}(P) = m. \]
\end{corollary}

Now consider that we want to count the number of eigenvalues of a matrix $A \in
\F^{n,n}$ in the interval~$[a, b]$.

\begin{theorem}[Schur form {\cite[Theorem~2.3.1]{Horn2012}}]
	Let $A \in \F^{n,n}$ have eigenvalues $\lambda_1, \lambda_2, \dotsc,
	\lambda_n$ in any prescribed order. Then there is a unitary matrix $U \in
	\F^{n,n}$ such that $T = U^* A U$ is an upper triangular matrix with
	diagonal entries $t_{ii} = \lambda_i$, $i = 1, 2, \dotsc, n$. If $A$ is real
	and has only real eigenvalues, then $U$ and $T$ can be taken to be real.
\end{theorem}

Let $U = [u_1, u_2, \dotsc, u_n]$ be a column partitioning of $U$. If
$\lambda_1, \lambda_2, \dotsc, \lambda_m$ are all the eigenvalues of $A$ in the
interval~$[a, b]$, then $Q \coloneqq [u_1, u_2, \dotsc, u_m]$ is an isometric
matrix, and $P \coloneqq Q Q^*$ is a projection with trace $m$. Clearly, to
estimate the number of eigenvalues in~$[a, b]$, we can approximate the trace of
$P$.



\section{Approximating An Invariant Subspace Projection}
\label{sec:projection-approximation}

Previously, we showed that we can count the number of eigenvalues in a given
interval~$[a, b]$ of a matrix $A$ by computing the trace of a projection onto
the invariant subspace corresponding to the eigenvalues in~$[a, b]$. In this
section we discuss how we can compute the projection. Obviously, we can
calculate the Schur decomposition of a $A$ to acquire $P$ but when we have the
Schur decomposition, the use of the projection is superfluous. Instead, we can
approximate the projection with polynomials.

Let $h(t): \R \rightarrow \R$ with
\[
	h(t) =
	\begin{dcases*}
		1 & if $t \in [a, b]$, \\
		0 & otherwise.
	\end{dcases*}
\]
If $p(t)$ is polynomial approximation to $h(t)$, then $p(A)$ will be an
approximation to the desired projection $P$. In this article, we will use
Chebyshev polynomials.

\begin{definition}[Chebyshev polynomial]
	The function $T_k: [-1, +1] \rightarrow \R$, where
	\[
		T_k(t) =
		\begin{dcases*}
			0 & if $k = 0$, \\
			t & if $k = 1$, \\
			2 t \, T_{k-1}(t) - T_{k-2}(t) & otherwise,
		\end{dcases*}
	\]
	is called \emph{Chebyshev polynomial} (of the first kind).
\end{definition}

Chebyshev polynomials are orthogonal polynomials.

\begin{theorem}
	It holds that
	\[
		\int\limits_{-1}^{+1} \frac{T_n(t) T_m(t)}{\sqrt{1 - t^2}}\,\mathrm{d}t=
		\begin{dcases*}
			\pi & if $n = m = 0$, \\
			\sfrac{\pi}{2} & if $n = m \neq 0$, \\
			0 & if $n \neq m$.
		\end{dcases*}
	\]
\end{theorem}

Let $d$ be the degree of the polynomial approximating $h(t)$. We can approximate
$h(t)$ with a linear combination of $d$ Chebyshev polynomials such that
\[ h(t) \approx \sum_{k=0}^d \gamma_k T_k(t), \]
where $\gamma_k \in \R$. The coefficients can be computed with the following
expression:
\[
	\gamma_k =
	\int\limits_{-1}^{+1} \frac{h(t) T_k(t)}{\sqrt{1 - t^2}}\,\mathrm{d}t,
	\, k = 0, 1, \dotsc, d.
\]
For the step function $h(t)$, the coefficients are \cite{DiNapoli2016}
\[
	\gamma_k =
	\begin{dcases*}
		\frac{1}{\pi} (\arccos(a) - \arccos(b)) & if $k = 0$, \\
		\frac{2}{\pi} \frac{\sin(k \arccos a) - \sin(k \arccos b)}{k} &
			if $k > 0$.
	\end{dcases*}
\]
Note that the polynomial $T_k$ given above is not normal and this fact was
incorporated into the coefficients. We can avoid the Gibbs oscillations with the
following Jackson coefficients \cite[§II.C.3]{Weisse2006}:
\[
	g_k^d =
	\frac{1}{d+1}
	\left[
		(d - k + 1) \cos\left(k \frac{\pi}{d+1}\right) +
		\sin\left(k \frac{\pi}{d+1}\right) \cot\left(\frac{\pi}{d+1}\right)
	\right], \, k = 0, 1, \dotsc, d.
\]
With these coefficients the approximating polynomial $p(t)$ is
\[ p(t) = \sum_{k=1}^d g_k^d \gamma_k T_k(t). \]
In Figure~\ref{fig:chebyshev-vs-jackson}, we plotted the Chebyshev and the
Chebyshev-Jackson approximations to the step function $h(t)$ with $a = -0.5$, $b
= +0.5$ with polynomials of degree 50 (cf.~\cite[Fig.~1]{DiNapoli2016}).

\begin{figure}
	\begin{center}
		\includegraphics{chebyshev-vs-jackson}
	\end{center}
	\caption{Approximation of the step function $h(t)$ ($a = -0.5$, $b =
		+0.5$) with a Chebyshev and a Chebyshev-Jackson polynomial of
	degree~50}
	\label{fig:chebyshev-vs-jackson}
\end{figure}

In order to approximate the desired projection, we have to compute
\[ P \approx \sum_{k=1}^d g_k^d \gamma_k T_k(A). \]



\section{Estimating the Matrix Trace}
\label{sec:estimating-matrix-trace}

So far, we found out that we can approximate the number of eigenvalues in an
interval~$[a, b]$ of a matrix with real eigenvalues $A$ by calculating the trace
of a suitable projection $P$. Also, we determined how to approximate $P$ by
means of a matrix polynomial. $P$ and $A$ are both $n \times n$ matrices so if
$n$ is large, we would like to avoid the construction of $P$ (or an
approximation thereof). In this section, we discuss matrix-free trace
estimators.

In general, we can compute estimates of the trace by Monte Carlo methods,
i.\,e., by calculating $v^* A v$, where $v \in \F^n$ is random. Naturally, we
can get a more accurate guess by averaging over $m$ random vectors such that
$T_m(A) \coloneqq \sfrac{1}{m} \sum_{i=1}^m v_i^* A v_i$. Ideally, $T_m(A)$ is
unbiased, has low variance, and is an $(\varepsilon, \delta)$-approximator:
given $\varepsilon > 0$, $0 < \delta \leq 1$, the probability that the relative
error of $T_m(A)$ is greater than $\varepsilon$ is smaller than $\delta$ or
equivalently,
\[
	\mathbb{P}(
		\abs{T_m(A) - \operatorname{tr}(A)} \leq
		\varepsilon \operatorname{tr}(A) )
	\geq 1 - \delta,
\]
where $\mathbb{P}(X)$ denotes the probability of an event $X$.

There are several popular distributions for the trial vectors or the entries of
the trial vectors, respectively (see, e.\,g., \cite[§3]{Avron2011}). In this
paper, we will focus on trial vectors with entries from the Rademacher
distribution (values $-1$ or $+1$ with probability $\sfrac{1}{2}$) or the
Gaussian distribution (with mean zero). For both distributions, $T_m(A)$ is
unbiased, the variance is known, and $T_m(A)$ will be an $(\varepsilon,
\delta)$-approximator. With Rademacher random variables, $T_m(A)$ is known as
\emph{Hutchinson estimator} \cite{Hutchinson1990}. Hence, we will denote this
estimator with $T_m^H(A)$ henceforth. For a single sample, the variance is
\[
	\mathbb{V}(T_1^H(A)) =
	2 \left( \norm{A}_F^2 - \sum_{i=1}^n a_{ii}^2 \right),
\]
where $\mathbb{V}(X)$ is the variance of the randomly distributed variable $X$.
For fixed $\varepsilon$ and $\delta$, the Hutchinson estimator is an
$(\varepsilon, \delta)$-estimator if
\[ m \geq 6 \frac{1}{\varepsilon^2} \ln \frac{2}{\delta}. \]
For the Gaussian distribution with mean zero, we denote the corresponding
estimator with $T_m^G(A)$. The variance of a single sample is
\[ \mathbb{V}(T_1^G(A)) = 2 \norm{A}_F^2 \]
and $T_m^G(A)$ is an $(\varepsilon, \delta)$-estimator if
\[ m \geq 8 \frac{1}{\varepsilon^2} \ln \frac{2}{\delta}. \]
(See \cite[Table~1]{Avron2011} for the variance, \cite{Roosta-Khorasani2015} for
the $(\varepsilon, \delta)$-estimator bounds.)
Furthermore, the following important theorem holds if $A$ is a projection.

\begin{theorem}[{\cite[Corollary~4]{Roosta-Khorasani2015}}]
	Let $P \in \F^{n,n}$ be a non-zero projection, let $[\cdot]$ denote the
	nearest integer function, and let $0 < \delta \leq 1$. If
	\[ m \geq 8 \rank(P) \ln \frac{2}{\delta}, \]
	then
	\[ \mathbb{P}\left( [T_m^G(P)] \neq \rank(P) \right) \leq \delta. \]
\end{theorem}

In practice, the bounds can be very pessimistic and the difference between
$T_m^G(A)$ and $T_m^H(A)$ is small in the experiments in
\cite[§5]{Roosta-Khorasani2015}, \cite[§9]{Avron2011}. To determine $m$, we can
compute Gaussian statistics (mean, variance) at run-time and increase $m$ until
the variance is sufficiently small. Let $\mu_k$, $\sigma_k$ be mean and standard
deviation of values $x_1, x_2, \dotsc, x_k$. Then it holds that
\begin{align*}
	\mu_{k+1} &= \frac{k}{k+1} \mu_k + x_{k+1}, \\
	\sigma_{k+1}^2 &=
		\frac{k (\sigma_k^2 + \mu_k^2) + x_{k+1}^2}{k+1} - \mu_{k+1}^2,
\end{align*}
where $\mu_1 = x_1$, $\sigma_1 = 0$.

There is also a Monte Carlo method based on subspace iterations that possesses
great convergence properties for projections because all eigenvalues are either
zero or one \cite{Saibaba2016}. This method cannot be used here because we would
need to know the algebraic multiplicity of the eigenvalue one in advance but
this is the quantity that we actually want to determine.



\section{Estimating Eigenvalue Counts for Generalized Eigenvalue Problems}

From the previous sections, we know how to estimate the number of eigenvalues in
the interval $[a, b]$ of a matrix $A$ with real eigenvalues, $-1 \leq a < b \leq
+1$. In this section, we will extend these methods to a Hermitian matrix pencil
$(K, M)$ with arbitrary real non-negative eigenvalues. Also, we will incorporate
the multilevel structure of the DCGeig generalized eigenvalue problem solver.

Let $K, M \in \F^{n,n}$, where $K$ is Hermitian positive definite and $M$ is
Hermitian positive semidefinite. Since $M$ may be singular, we will consider
eigenvalue counts based on $A \coloneqq K^{-1} M$ and with this provision, we
can we can explain how to count eigenvalues. Let $\mu_1 \geq \mu_2 \geq \dotsb
\geq \mu_n \geq 0$ be the eigenvalues of $M x = \mu K x$. In order to count
eigenvalues in the interval $[a, b]$, we have to transform the spectrum of $(M,
K)$ such that the transformed eigenvalues are in $[-1, +1]$. For the generalized
eigenvalue problem $(K, M)$, we are interested in the eigenvalues $\lambda \leq
\lambda_c$ or equivalently, the largest eigenvalues of $(M, K)$. Therefore, $a =
1/\lambda_c$, $b = \mu_1$. Furthermore, all eigenvalues are non-negative so we
can simply assume $\mu_n = 0$ and with this assumption, we do not have to
compute (or approximate) $\mu_n$. Thus, let
\[ c = \frac{\mu_1}{2}, e = \frac{2}{\mu_1}, \]
then the matrix pencil $(\sfrac{1}{e} (M-c), \sfrac{1}{e} K)$ has only
eigenvalues in $[-1, +1]$ and we want to count the number of eigenvalues of this
matrix pencil in $[\sfrac{a-c}{e}, 1]$. Observe that the right-hand of the
interval is one. This means, the filter polynomial only has to approximate the
the heaviside function and we can use lower-degree polynomial. The coefficients
for the filter polynomial can be found in
Section~\ref{sec:projection-approximation} and with the techniques from
Section~\ref{sec:estimating-matrix-trace} we can estimate the number of
eigenvalues $\lambda \leq \lambda_c$ as soon as we acquired a good estimate of
the smallest eigenvalue of $(K, M)$.

Reconsider the $2 \times 2$ block partitioning of $(K, M)$:
\[
	\begin{pmatrix} K_{11} & K_{12} \\ K_{21} & K_{22} \end{pmatrix}
	x
	= \lambda \begin{pmatrix} K_{11} & K_{12} \\ K_{21} & K_{22} \end{pmatrix}x.
\]
Let $n_c$ be the number of eigenvalues $\lambda \leq \lambda_c$, let $Q_i$, $i =
1, 2$, be isometric matrices such that $Q_i^* K Q_i = K_{ii}$, $Q_i^* M Q_i =
M_{ii}$, and let $X_c \in \F^{n,n_c}$ be the matrix of eigenvectors
corresponding to eigenvalues $\lambda \leq \lambda_c$. DCGeig is a multilevel
eigensolver; it computes recursively approximations to $X_c$. Consequently, we
are not only interested in the value $n_c$ but also in the size of the search
spaces calculated by $(K_{11}, M_{11})$, $(K_{22}, M_{22})$, i.\,e., the rank of
$Q_i^* X_c$, $i = 1, 2$. We can approximate these quantities with previously
computed projection $P$.

\begin{theorem}
	It holds that
	\[ \rank(Q_i^* X_c) = \rank(Q_i^* P), \, i = 1, 2. \]
\end{theorem}

\begin{proof}
	It holds that $\ran X_c = \ran P$. $Q_i$ has full column rank so
	consequently,
	\[ \ran(Q_i^* X_c) = \ran(Q_i^* P). \]
	Since $\rank A = \dim \ran A$, $A \in \F^{m,n}$, we have
	\[ \rank(Q_i^* X_c) = \rank(Q_i^* P). \]
\end{proof}

We can use this theorem to apply the methods of
Section~\ref{sec:estimating-matrix-trace} to estimate $\rank(Q_i^* X_c)$. Let
$n_i \in \mathbb{N}$, $i = 1, 2$, such that $Q_i \in \F^{n,n_i}$, and let $v_i
\in \F^{n_i}$. Then
\begin{align*}
	\rank P &\approx v^* p(K^{-1} M) v, \\
	\rank(Q_i^* P) &\approx v_i^* Q_i^* p(K^{-1} M) Q_i v_i, \, i = 1, 2.
\end{align*}

In any case, we have to compute $p(K^{-1} M)$ (instead of, say, $p(K_{ii}^{-1}
M_{ii})$) but on the upside, $n_c = n_1 + n_2$ so estimating $n_c$ is
superfluous after we estimated $n_1$, $n_2$.

\printbibliography

\end{document}
